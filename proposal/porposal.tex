\documentclass{article}
\usepackage{fancyvrb}
\usepackage{fancyhdr}
\usepackage{extramarks}
\usepackage{amsmath}
\usepackage{amsthm}
\usepackage{amsfonts}
\usepackage{tikz}
\usepackage{graphicx}
\usepackage{enumitem}
\usepackage[plain]{algorithm}
\usepackage{algpseudocode}
\usepackage[normalem]{ulem}
\usepackage{hyperref}
\usepackage{xcolor}
\usepackage{caption}
\usepackage{subcaption}
\usepackage{listings}% http://ctan.org/pkg/listings
\lstset{
  basicstyle=\ttfamily,
  mathescape
}

\usetikzlibrary{automata,positioning}

\topmargin=-0.45in
\evensidemargin=0in
\oddsidemargin=0in
\textwidth=6.5in
\textheight=9.0in
\headsep=0.25in

\linespread{1.1}

\pagestyle{fancy}
\lhead{\hmwkAuthorName}
\rhead{\hmwkClass\ (\hmwkClassInstructor\ \hmwkClassTime): \hmwkTitle}
% \rhead{\firstxmark}
\lfoot{\lastxmark}
\cfoot{\thepage}

\renewcommand\headrulewidth{0.4pt}
\renewcommand\footrulewidth{0.4pt}

\setlength\parindent{0pt}

\setcounter{secnumdepth}{0}
\newenvironment{homeworkProblem}[1][-1]{
    \ifnum#1>0
        \setcounter{homeworkProblemCounter}{#1}
    \fi
    \section{Problem \arabic{homeworkProblemCounter}}
    \setcounter{partCounter}{1}
    \enterProblemHeader{homeworkProblemCounter}
}{
    \exitProblemHeader{homeworkProblemCounter}
}

\newcommand{\hmwkTitle}{FP Proposal}
\newcommand{\hmwkClass}{CSE 512}
\newcommand{\hmwkClassTime}{Lecture A}
\newcommand{\hmwkClassInstructor}{Jeffrey Heer}
\newcommand{\hmwkAuthorName}{Emily Gu | Ryan Drapeau | Vimala Jampala}

\title{
    \vspace{2in}
    \textmd{\textbf{\hmwkClass:\ \hmwkTitle}}\\
    \vspace{0.2in}\large{\textit{\hmwkClassInstructor\ \hmwkClassTime}}\\
    \author{\textbf{\hmwkAuthorName\ $\vert$ \hmwkAuthorCSE\ $\vert$ \hmwkAuthorId}}
}

\date{}

\begin{document}

\begin{center}
\LARGE
\textbf{Final Project Proposal}
\end{center}

We plan to continue development on our Assignment 3 project, CourseRatings, which visualizes the Course Evaluation Catalog in a more intuitive and user-friendly way. Currently, CourseRatings provides the option to search and sort through the vast amounts of data. You can also drill down into a specific course or instructor to limit the results to a respective category. There are many improvements that could be made to turn this into a good final project. We plan to use D3 to add visualizations of the ratings over time to each course and instructor page. Visualizations could also be added to department pages. This would allow for a more detailed breakdown of courses and instructors.  Additionally, CourseRatings initially starts with an empty page. We plan to use D3 to add visualizations depicting interesting trends across our dataset to this page. For example, we could display which departments have the easiest classes, which professors have the highest overall rating, and so on. We also have the quarter the course was taught so we could also show visualizations of the ratings over time.\\

There are also many different improvements that have been suggested by various people in the class. The most common suggestion was to implement auto-prediction of search queries. Another common request was to incorporate a tutorial or example of how to use all of the features in the application. There are also features that we thought of as we neared the end of project 3. We have the percentage of students that filled out each survey for every course. We plan to explore the idea of weighting ratings by the percentage of students in the class that filled out the survey because a higher percentage results in a more accurate response.\\

Links:
\begin{enumerate}
    \item {\color{blue} \href{https://www.washington.edu/cec/toc.html}{Course Evaluation Catalog}}
    \item {\color{blue} \href{https://github.com/CSE512-15S/a3-emilygu-drapeau-vjampala}{Project 3}}
    \item {\color{blue} \href{https://students.washington.edu/drapeau/course_ratings/}{CourseRatings}}
\end{enumerate}

\end{document}
